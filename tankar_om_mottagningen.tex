\documentclass[a4paper]{article}

\usepackage[english]{babel}
\usepackage[utf8]{inputenc}
\usepackage{graphicx, float}
\usepackage{url}

\title{Tankar och ideér gällande rekå och mottagningen 
i allmänhet. }

\author{Marcus Granberg}

\date{\today}

\begin{document}

\maketitle

\newpage

\section{Random shit}

Är ett tag kvar tills det här ska skickas in så börja 
bara skissa på hur det ska se ut. Gör ett skelett. Vad 
ska skrivas var och hur. etc etc \\ \\ 
Börja med att presentera dig. Namn, ålder, program, 
familj etc etc. Skriv kortfattat ner vad du gillar, 
vad som definierar dig och typ det som poppar fram när 
du brainstormar i det är stadiet. Den här biten är 
väldigt simpel. Behöver inte direkt göra några 
efterforskningar. BESKRIV VEM DU ÄR! Den här biten ger 
ett bra första intryck på vem du är men nästa stycke 
är nästan ännu viktigare. \\ \\
Nästa stycke ska handla om varför just jag Marcus 
Granberg är den de behöver till posten jag söker. Här 
behövs det en del efterforskning. Dvs läsa igenom 
dokument. Börja med de gamla dokumenten och sen när de 
nya uppdaterade kommer upp så fokusera på dem 
istället. De kommer att vara liknande om det inte har 
skett något väldigt revolutionerande. Läs igenom 
uppdragsbeskriningen för hela mottagningen 
\url{file:///C:/Users/Marcus/Downloads/Uppdragsbeskrivning%20-%20Teknolog-%20och%20datavetarmottagningen%202016-160909.pdf}.
Det kommer att ge dig mer kött på benen när de ställer 
"viktiga" frågor. Samma sak med alla 
postbeskrivningar. Se till att ha läst igenom alla 
poster inom rekå, så att du har koll på vilka som 
tillhör vilken grupp och om man kan utöka sammarbete 
eller något liknande. Gör detta även fast du inte är 
intresserad av sagd post. \\ \\
Det viktigaste med att läsa igenom alla posterna är 
att bestämma sig vilken post du vill söka FÖR JAG HAR 
INTE BESTÄMT MIG. För att hitta alla postbeskrivningar 
så använd denna URL 
\url{https://files.utn.se/public.php?service=files&t=336f9da5a7169c41de99dd7286b6dee1&path=%2FUppdrags-%2C%20Befattnings-%20och%20Postbeskrivningar%2FTD-mottagningen%2FReccentiorkommitt%C3%A9en}

\section{Uppdragsbeskrivning för Uppsala teknolog- och 
naturvetarkårs teknolog- och datavetarmottagning}

Detta dokument är det inte så mycket att prata om. Det 
beskriver TD-mottagningen i helhet och diskuterar 
vissa frågor. Det som står i dokumentet är bra att 
veta, men mycket av det rör till större del de som 
sitter på presidieposterna. Kortfattat, jag tror att 
jag inte behöver skriva något angående dokumentet i 
ansökan men det kan vara bra att läsa igenom det 1-2 
ggr till så att man inte missar någon information de 
skulle vilja diskutera på en möjlig intervju.

\section{Poster inom Rekå}

Här under kommer jag att prata och diskutera angående 
de olika posterna inom Rekå. Vad de handlar om, vilken 
grupp de tillhör och om jag är intresserad av sagd 
post.

\subsection{Binär 2016}

Som Binär har man ansvar för TD-mottagningens hemsida 
och samtliga mailkonton inom TD-mottagningen. Ansvarar 
för marknadsföring som sker via sociala medier. 
Ansvarar även för databasen över faddrar, reccar och 
volontärer och slutligen anmälnings- och 
uppdragsutdelningssystemen. 

Så kortfattat så har man huvudansvaret över "all" 
information som är itbaserad. 

\subsection{Binär 2017}



\subsection{Bookmaker 2016}



\subsection{Bookmaker 2017}

bookmaker ansvarar för recceboken och des innehåll. 
Detta inkluderar indrivning av interna text och bilder 
till recceboken. Är även med i Tryckkå. Utöver detta 
så ingår även allt det arbete som rör hela rekå, med 
planering och fixande.

\subsection{Bus 2016}



\subsection{Bus 2017}

Bus 1 och Bus 2 ansvarar för alla lekar och sånger för 
mottagningen. Detta inkluderar reccesång, reccedans, 
gasqueinfo, klassolumpiad, förköret till klassfesten 
samt intåget. Ser även till att spexövningar och 
liknande genomförs. Utöver detta så ska de hjälpa till 
med Rekås gemensamma åttaganden. 

\subsection{Fix 2016}



\subsection{Fix 2017}

Fix införskaffar/planerar tält och det som här där 
till. Dvs ljud ljus, vatten och avspärring. Fix ska 
även fundera på inhyrningen av tältet och 
ljudproblemet och DJs/underhållning. Tillsammans med 
skum så ska säkerhetsrutiner/tillstånd arbetas fram. 
Ska vara med i 
säkerhetskå/faddertackå/mörkå/finaldagskå/reccetältskå.  

\subsection{Kalaspinglan 2016}



\subsection{Kalaspinglan 2017}

Kalaspinglan ansvarar för klassfesterna (som 
inkluderar tåga, personal och garderob). Ansvarar även 
för de engångsartiklar som används i samband med 
mottagningen. Vidare så ansvarar kalaspinglan för 
bffts. Sitter i 7-år-senarekå, faddertackkå, ladankå, 
finaldagskå etc etc.

\subsection{Koll 2016}



\subsection{Koll 2017}

Koll ansvarar för schemat för reccarna, lokalbokning 
och lokalaccess. Utöver detta så är koll även ansvarig 
för planeringen av uppropen och samordningen mellan TD 
och universitetet. Ingår i Schemakå.

\subsection{Krams 2016}



\subsection{Krams 2017}

Krams ansvarar för att lägga internschema för rekå. 
Krams ansarar även för fakematten, samvetscentralen, 
samvetstillverkning och lokalvård. Ingår i schemakå.

\subsection{Köx 2016}

Köx ansvarar för det som rör kök och livsmedel under 
mottagningen m.m. Ansvarar för att det ska finnas 
tydliga rutiner för hur mat hanteras och så att alla 
som hanterar mat vet vilka regler som gäller. Dvs 
utbilda TD och resten i hur det ska fungera. Hjälpa 
kökspersonalen så de vet vad de ska göra. Ska ingå i 
Faddertackkå, Mediapubkå, reccetältkå. 

\subsection{Köx 2017}



\subsection{Skum 2016}



\subsection{Skum 2017}

Skum ansvarar för allt som har med dryck att göra 
(sammarbeta med UTNs klubbmästare). Ansvarar för baren 
i tältet och de rutiner som rör detta. Vidare så ska 
skum planera brand och förstahjälpen utbildningen. 
sammarbetar med fix med säkerhetsrutiner och tillstånd 
under arrangemanget. Ingår i säkerhetskå, mediapubkå, 
faddertackkå. 

\subsection{Spons 2016}



\subsection{Spons 2017}

Spons tar hand om majoriteten av sponsringen. De 
ansvarar även för mottagningsmässan och lunchföredrag 
som förekommer under mottagningen. Vidare har de hand 
om insponsring av bilar och partykit. Kortfattat så 
har de huvudansvar om i princip all sponsring till 
mottagningen. 

\subsection{Taesk 2016}



\subsection{Taesk 2017}

Taesk ansvarar för City-goes-crazy och 
Multi[Taesk]games. Detta inkluderar utvärdering, 
utformning och genomförande. Utöver detta så ska även 
Taesk ansvara för utformningen av klassuppdragen och 
Recceenkäten. I år så har även Taesk ansvar för 
utvärdering och genomförande av uppladdningen. 

\subsection{Krims 2016}



\subsection{Krims 2017}

Krims ansvarar för kontakten mellan TD och nationerna, 
samt planeringen av reccegasquen tillsammans med 
Sting. Krims ska hålla kontakt med Fadderkå angående 
måndags och lördagsfesten. Nationskontakten inkluderar 
inskrivningen på nationerna och medlemsrekrytering 
till UTN. Krims rekryterar även de volontärer som 
behövs och det som hör till volontärerna. Ansvarar 
även för finaldagens utformning. Ingår i Nationskå, 
Volliskå och finaldagskå.

\subsection{Sting 2016}



\subsection{Sting 2017}

Sting ansvarar för Stingfestivalen och tillhörande 
aktiviteter. Detta inkluderar även kontakten med 
föreningarna på campus, samt indrivningen av deras 
texter till recceboken och hemsidan. Ingår i Anhörigkå 
och Nationskå.

\section{Årets fokusfråga}

Inför årets mottagning ligger fokus på att utvärdera 
och se över mottagningens struktur och organisation 
(arbetsbörda, ansvarsfördelning, innehåll) för att 
utveckla mottagningen på ett hållbart sätt inför 
framtiden. Detta innebär även att arbeta för en mer 
enad mottagning. Förändringar som gjordes förra året 
kommer att utvärderas och vidareutvecklas.

Så kortfattat vad kan man göra om eller förbättra med 
din tänkta post och med mottagningen/grupperna i 
allmänhet.

Ska utvecklas

Binär: All kod ska ha beskrivningar etc etc.

Alla: Arbetsbörda. Rent textmässigt så ser det lite 
ojämt ut men det är något man måste kolla på. 


\section{Slutsats}



\end{document}

